At the start of our work we inherited a large dataset from Rejaie \textit{et all} that was created through a biased sampling of the Twitter network.  The dataset contained partial connectivity information for 242,275 potentially influential users.  Due to limitations of the Twitter REST API at the time of collection, the connectivity data was incomplete.  Additionally, the dataset did not include the associated attribute information for each user because it was not germaine to Rejaie's previous work.  This dataset needed to be completed before beginning our analysis.

Our goal was to characterize the relationships in a completed section of the Twitter network.  It was first necessary to determine if a large, complete subgraph existed within this dataset.  A collection of smaller subgraphs would not be sufficient as they would be less likely to capture the impact of rare influential users.  By performing a series of reconstructions of the graphs in the dataset we were able to discover a subgraph of 215,606 sampled users.  Including the next hop information for these sampled users, the subgraph included 15,548,091 unique user accounts.

The inherited dataset was not originaly intended to measure the extreme connectivity degrees of rare users.  As a result, the friend and follower counts for each user in the subgraph were truncated at 1,000.  This truncation was due to the prohibitive rate limiting of of the Twitter API, which made it infeasible to collect the complete connectivity information for these users in real time.  Over the course of several weeks, we were able to restore this missing information.  Several months passed between these two collection phases.  The potential error that this introduced is explored below.

Simultaneously, we needed to collect the user attribute information for the tens of millions of users in our subgraph.  Completing this task via the Twitter API was difficult due to the connectivity restoration process, and it was not our wish to violate Twitter's terms of service by launching API calls from dozens of hosts.  Fortunately, we were able to find an alternate method of collecting user attribute information.  Twitter provides XML dumps of user information through their primary website.  These calls are not subject to API rate limites, presumably for the benefit of mobile applications that may issue frequent user lookups.  Using this service, we were able to collect the necessary information without violating the Twitter API ToS.

\subsection{Measuring Error}

Because our user attribute information was collected in less than a month, we have assumed that no significant error is present.  It is not possible to capture a realtime snapshot of user attributes in the Twitter network, so we have no method of comparing collected attributes to actual attributes.  Our intuition here is that most user attributes are not subject to frequent change.  The exception to this is friend and follower count, which is explored below.  For our purposes, the collected user attribute information can be accepted as accurate.

In contrast, connectivity information for our subgraph was collected in two different phases between September 2010 and February 2011.  It would be foolish to assume that significant change did not occur in the network over this time.  It was therefore necessary to confirm that the temporally disparate collection phases did not introduce error to our network measurements.  To do this, we compared our snapshot of connectivity for each user to the stated friend and follower counts in each user's attributes.

GRAPHIC for FOLLOWERS: The CDF I am working on that shows the error percentage for our core users between actual degree and measured degree.\\

GRAPHIC for FRIENDS: The CDF I am working on that shows the error percentage for our core users between actual degree and measured degree.\\

Figures \ref{fo-cdr} and \ref{fr-cdr} confirm that we have captured an accurate view of connectivity in our piece of the Twitter network.  90\% of users blah blah blah.  This speaks to the general stability of the social network that Twitter facilitates.

(Q: Plotting the growth of this subgraph -- would this be a neat consideration for future work?  Not that we need to do it, but we can always suggest it in the paper for the benefit of Team Reza)\\

\subsection{Profiling}

Paragraph: Restate what we said in the introduction about how we need to compare our subgraph to a random subgraph in that magical way that papers manage to do.

GRAPHICS FOR EACH ATTRIBUTE!\\

Paragraphs summarizing the attributes.\\

Paragraph: Conclude that our dataset is sufficiently representative of a random twitter sampling to warrant further investigation.  Make note of any attributes for which this is not the case.\\

