\subsection{Approaching Attribute Analysis}

Paragraph: Describe the manner in which we approach each attribute.  Explain that for some attributes with a small number of options it is easy to investigate all simultaneously.  Explain for the range options that we created logarithmic bins to accurately capture our extreme uses.\\

Paragraph: Explain how Location was an entirely different animal because it is a freely entered text field.  A large number of users do not put valid locations in this field.  For the users that do, those options are often difficult to parse.\\

Paragraph:  Mention that some attributes were not considered at all.  Explain the attributes that were not considered because they were completely irrelevant (bg color)  Explain the attributes that were not considered because their values were not sufficiently heterogenous (verified).\\

Paragraph: Mention the simple random survey of records that was performed, the percentage of users with no entry, the percentage of users with an indecipherable entry, etc.\\

Paragraph: Explain that it was decided to evaluate location based on state.\\

\subsection{Evaluation}

Paragraph: Explain the basic challenge of deriving bias from the measured following patterns.

Paragraph: Explain the method for calculating the expected connectivity in a randomized, bias-free version of our subgraph.  Include formula.

Paragraph: Explain the method for calculating bias from this value and the measured values.  Include formula.

