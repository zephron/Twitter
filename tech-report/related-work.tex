Our analysis of Twitter is informed by the previous work of Rejaie \textit{et all} on the characterization of overlays in P2P file-sharing systems.  This analysis of the Gnutella network discovered a bias in peer connectivity based on age in the system, creating an onion-like overlay in which the longest active peers formed the center of the network~\cite{RS05}.  Our work borrows this concept of connectivity bias and applies it to the wide variety of attributes present in the Twitter system.  The challenges in system characterization vary between P2P networks and online social networks.  One example of this is volatility of the network.  Twitter's structure is much more stable than Gnutella's, although it is still difficult to acquire an accurate snapshot of the network due to Twitter's rate limits on access to network measurement tools.\\
One of Twitter's core functions is the analysis of individual messages to determine trending topics.  As such, Twitter is capable of acting a cultural barometer.  A variety of studies have attempted to answer the question of how information spreads within the Twitter network.  Investigations have included spam, the meme life cycle, and news discovery and explanation.\\
A variety of Twitter-related work has been conducted over the past several years.  The majority of this work focuses on the analysis and propagation of individual messages as they spread through the network.  Many have characterized the nature of spam on Twitter~\cite{GTPS10}~\cite{Yardi_Romero_Schoenebeck_Boyd_2010}~\cite{DBLP:journals/corr/abs-1011-3768}.  Cheng \textit{et al} have used tweets to attempt to geo-locate users based on the content of their messages~\cite{CCL10}.  Others have attemped to aggregate message feeds in order to discover real world events~\cite{Lee:2010:MGR:1867699.1867701}~\cite{Fujisaka:2010:MGA:1880853.1880896}.  Lerman and Ghosh have studied Digg and Twitter to measure news items' lifespan and speed within social networks~\cite{LG10}.  Sadikov and Martinez have conducted similar work, chosing instead to focus on URL and tag propagation~\cite{SM09}.\\
It is only reasonable that message-level analysis has captured the attention of the research community.  After all, the novelty of modern social networks is largely that they are a new medium for the spread of information.  For many purposes, however, message-level analysis does not tell the whole story.  Twitter is a networked system of users exchanging information.  Message-level work has frequently relied on predictive message filtering or spotting trending phrases.  This is not perfectly suited for the task of broadly assessing the state of the network.  Instead, we ask a much simpler question -- what type of people are different users choosing to listen to?\\
For this, we have turned to user-level analysis.  When a user choses to opt-in to another user's tweet stream, this says much more about influence than the propagation of individual messages.  This idea of influence through followers is a truth that rests at the very core of Twitter, one that can be plainly seen by visiting any user's page and making note of the prominently displayed \textit{Follower} and \textit{Following} counts.  Here, we taken this fundemental concept and cross-reference it with various user attributes in an attempt to make more general claims about the nature of influence in the Twitter network.