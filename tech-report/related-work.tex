One of Twitter's core functions is the analysis of individual messages to determine trending topics.  As such, Twitter is capable of acting a cultural barometer.  A variety of studies have attempted to answer the question of how information spreads within the Twitter network.  Investigations have included spam, the meme life cycle, and news discovery and explanation.\\\\
A variety of Twitter-related work has been conducted over the past several years.  The majority of this work focuses on the analysis and propagation of individual messages as they spread through the network.  Cheng \textit{et al} have used tweets to attempt to geo-locate users based on the content of their messages ~\cite{CCL10}.  Lerman and Ghosh have studied Digg and Twitter to measure news items' lifespan and speed within social networks ~\cite{LG10}.  Sadikov and Martinez have conducted similar work, chosing instead to focus on URL and tag propagation ~\cite{SM09}.\\
It is only reasonable that message-level analysis has captured the attention of the research community.  After all, the novelty of modern social networks is largely that they are a new medium for the spread of information.  For the purposes of assessing influence, however, message-level analysis does not tell the whole story.  Much of this work lies in predictive message filtering or spotting trending phrases.  This is not perfectly suited for the task of broadly determining who is influencing who.  Instead, we ask a much simpler question -- what type of people are users choosing to listen to?\\
For this, we turn to user-level analysis.  When a user choses to opt-in to another user's tweet stream, this says much more about influence than the propagation of individual messages.  This idea of influence through followers is a truth that rests at the very core of Twitter, one that can be plainly seen by visiting any user's page and making note of the prominently displayed ``Follower'' and ``Following'' numbers.  Our contribution will be to take this fundemental concept and cross-reference it with various user attributes in an attempt to make more general claims about the nature of influence in the Twitter network.\\